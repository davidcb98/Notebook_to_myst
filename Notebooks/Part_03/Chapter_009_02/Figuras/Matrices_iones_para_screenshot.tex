\documentclass[a4paper,11pt]{article} % Tamaño de papel, tamaño de letra y clase LaTeX usada...

\usepackage[spanish]{babel} % Permite escribir en Castellano (tildes y ñ) e implementa...
\usepackage[utf8]{inputenc} % Permite introducir acentos y Ñ directamente.
%\usepackage[T1]{fontenc} %permite cambiar la fuente por defecto.
\usepackage{bm} % Permite letras griegas en negrita {uso: \bm{\alpha}}
\usepackage{graphicx}
\usepackage{amssymb}
\usepackage{mathtools}
\usepackage{subfigure}
\usepackage{multicol} %para añadir texto en columnas
\usepackage{float}
\usepackage{fancyhdr} %encabezados y pies de pagina
\usepackage{amsmath,amsfonts,latexsym,color,textcomp,anysize}

\usepackage[table,xcdraw]{xcolor}
\usepackage{longtable}
\usepackage{parskip}
\usepackage{hyperref}
\usepackage{cancel} % tachar cosas \cancel{}
%\usepackage{tensor} % Notación: \tensor{T}{^a_b^c_d}
%\usepackage{mathrsfs}
\usepackage{slashed}   %\slashed{p}
\spanishdecimal{.}
\usepackage{cite}
%\usepackage[style=numeric-comp]{biblatex}
\usepackage{yhmath}

\hypersetup{
    colorlinks=true,
    linkcolor=blue,
    filecolor=magenta,      
    urlcolor=cyan}

\usepackage[braket, qm]{qcircuit}

\setcounter{MaxMatrixCols}{12}
\usepackage{changepage}
\usepackage{framed}
\usepackage{caption}
%\usepackage{subcaption}

% =========================================================================================
% Cuadros
\usepackage[most]{tcolorbox}
\tcbuselibrary{listingsutf8}
\newtcolorbox{mybox_blue}[1]
{enhanced jigsaw, breakable, pad at break*=1mm,  colback=cyan!5!white,colframe=cyan!75!black,fonttitle=\bfseries,title=#1}

\newtcolorbox{mybox_green}[1]
{enhanced jigsaw, breakable, pad at break*=1mm, colback=green!5!white,colframe=green!45!black,fonttitle=\bfseries,title=#1}

\newtcolorbox{mybox_red}[1]
{enhanced jigsaw, breakable, pad at break*=1mm, colback=red!5!white,colframe=red!45!black,fonttitle=\bfseries,title=#1}

\newtcolorbox{mybox_orange}[1]
{enhanced jigsaw, breakable, pad at break*=1mm, colback=orange!10!white,colframe=orange!80!black,fonttitle=\bfseries,title=#1}

\newtcolorbox{mybox_gray}[1]
{enhanced jigsaw, breakable, pad at break*=1mm, colframe=gray!45!black}

\newtcolorbox{mybox_gray2}[1] % Sin bordes
{enhanced, sharp corners, breakable, pad at break*=1mm, boxrule=0pt, toprule=1pt, bottomrule=1pt, colframe=black}




% =========================================================================================
% codigo Python
\usepackage{listings} 	% \begin{lstlisting}
\definecolor{dkgreen}{rgb}{0,0.6,0}
\definecolor{gray}{rgb}{0.5,0.5,0.5}
\definecolor{mauve}{rgb}{0.58,0,0.82}
\lstset{frame=tb,
  language=Python,
  aboveskip=3mm,
  belowskip=3mm,
  showstringspaces=false,
  columns=flexible,
  basicstyle={\small\ttfamily},
  numbers=none,
  numberstyle=\tiny\color{gray},
  keywordstyle=\color{blue},
  commentstyle=\color{dkgreen},
  stringstyle=\color{mauve},
  breaklines=true,
  breakatwhitespace=true,
  tabsize=3
}

%Includes "References" in the table of contents
\usepackage[nottoc]{tocbibind}

%===========================================================================================
% Margenes, sangria, espacio entre parágrafos

%\marginsize{2cm}{2cm}{2cm}{1.5cm} %MÁRGENES: Izq, Der, Sup, Inf.
\parindent=0mm % Sangría por defecto
\parskip=3mm % Espacio entre párrafos por defecto

%===========================================================================================
% Definiciones de utilidad

% Parentesis
\def\lp{\left(}
\def\rp{\right)}

\def\lc{\left[}
\def\rc{\right]}

\def\lch{\left\{}
\def\rch{\right\}}

\def\l|{\left|}
\def\r|{\right|}

\def\Lp{\Bigl(}
\def\Rp{\Bigr)}

\def\Lc{\Bigl[}
\def\Rc{\Bigr]}

\def\Lch{\Bigl\{}
\def\Rch{\Bigr\}}

\def\L.{\Bigl.}
\def\R.{\Bigr.}

% Cosas cuanticas
\newcommand{\branew}[1]{\langle #1|} 
\newcommand{\ketnew}[1]{|#1\rangle} 
\newcommand{\braket}[2]{\langle #1|#2\rangle} 
\newcommand{\ketbra}[2]{| #1\rangle \! \langle #2|} 
%\newcommand{\ketbra}[2]{| #1\rangle \langle #2|} 
\newcommand{\cg}[1]{{\rm C}#1} 

% Cosas útiles
\def\Nabla{\bm{\nabla}}
\def\rqa{\quad \Rightarrow \quad}
\def\senc{\, \text{senc}}
%===========================================================================================
% Para poner subsubsections en letra pequeña y crear subsubsubsection en letra pequeña

% Subsub en cursiva y subrayado
\def\subsubiContadorIt{\par\addtocounter{subsubsection}{1}\underline{\it\thesubsubsection.}\hskip0.5cm \setcounter{subsubsubsectionIt}{0}}
	\newcommand{\SubsubiIt}[1]{
		\subsubiContadorIt \textit{#1}
	}

% subsubsub con cursita y subrayado
\newcounter{subsubsubsectionIt}[subsubsection]
\def\subsubiiContadorIt{\par\addtocounter{subsubsubsectionIt}{1}\underline{\it \thesubsubsection.\thesubsubsubsectionIt.}\hskip0.5cm}
	\newcommand{\SubsubiiIt}[1]{
		\subsubiiContadorIt \textit{#1}
	}

% subsubsub negrita
\newcounter{subsubsubsectionBf}[subsubsection]\def\subsubiiContadorBf{\par\addtocounter{subsubsubsection_bf}{1}\bf \thesubsubsection.\thesubsubsubsection.\hskip0.5cm}
	\newcommand{\SubsubiiBf}[1]{
		\subsubsubsectionBf \textbf{#1}
	}
%===========================================================================================

%\title{\Huge{\textbf{Introducción a la Computación Cuántica.}}}
%\author{David Castaño Bandín (UMA)}
%\date{Año 2023}
%\newpage



% ================================= PORTADA =======================================
%\pagestyle{empty}

\begin{document}
\renewcommand{\tablename}{Tabla}

\SubsubiIt{Carrier resonance}


% \makebox[\widthof{$\ket{g,n}$}]{$0$}
% \makebox[\widthof{$\ket{g,n+1}$}]{$0$}  
% \makebox[\widthof{$\ket{e,n}$}]{$0$}
% \makebox[\widthof{$\ket{e,n+1}$}]{$0$}
Para este caso tenemos 
\begin{equation*} \label{ec_ions_U_carrier}
 \begin{matrix}
    & 
    %
    \begin{matrix}
      \makebox[\widthof{$-i e^{i \phi} \sin \lp \beta/2 \rp$}]{$\ket{g,n}$}   & 
      \ket{g,n+1} & 
      \makebox[\widthof{$-i e^{-i \phi} \sin \lp \beta/2 \rp$}]{$\ket{e,n}$}   &
      \ket{e,n+1}
    \end{matrix} 	&  \\[.5\normalbaselineskip]
  %
    \mathcal{U}_I^{(\text{c})} = & 
    \begin{bmatrix}
        \cos \lp \beta/2 \rp  & 
        \makebox[\widthof{$\ket{g,n+1}$}]{$0$}   & 
        -i e^{-i \phi} \sin \lp \beta/2 \rp & 
        \makebox[\widthof{$\ket{e,n+1}$}]{$0$} \\
      %
        0   & 
        \makebox[\widthof{$\ket{g,n+1}$}]{$1$}   & 
        0 & 
        \makebox[\widthof{$\ket{e,n+1}$}]{$0$} \\
      %
        -i e^{i \phi} \sin \lp \beta/2 \rp & 
        \makebox[\widthof{$\ket{g,n+1}$}]{$0$}   & 
        \cos \lp \beta/2 \rp & 
        \makebox[\widthof{$\ket{e,n+1}$}]{$0$} \\
      %
        0 & 
        \makebox[\widthof{$\ket{g,n+1}$}]{$0$}   & 
        0 & 
        \makebox[\widthof{$\ket{e,n+1}$}]{$1$} 
    \end{bmatrix} &
    \begin{matrix} \ket{g,n} \\ \ket{g,n+1} \\ \ket{e,n} \\ \ket{e,n+1} \end{matrix} 
\end{matrix} \,
\end{equation*}

donde $\phi$ es la \textbf{fase} del pulso láser y $\beta = \Omega t$, siendo $\Omega$ la intensidad de acoplamiento (un parámetro) y $t$ el \textbf{tiempo} de aplicación del pulso láser. Para nuestro caso, $n = 0$. 


		\SubsubiIt{Blue sideband}
		
Para este caso tenemos 
\begin{equation*} \label{ec_ions_U_bsb}
 \begin{matrix}
    & 
    %
    \begin{matrix}
      \makebox[\widthof{$-i e^{i (\phi + \pi/2)} \sin \lp \eta\beta/2 \rp$}]{$\ket{g,n}$}   & 
      \ket{g,n+1} & 
      \ket{e,n}   &
      \makebox[\widthof{$-i e^{-i (\phi+ \pi/2)} \sin \lp \eta\beta/2 \rp$}]{$\ket{e,n+1}$} 
    \end{matrix} 	&  \\[.5\normalbaselineskip]
  %
    \mathcal{U}_I^{(\text{bsb})} = & 
    \begin{bmatrix}
        \cos \lp \eta\beta/2 \rp  & 
        \makebox[\widthof{$\ket{g,n+1}$}]{$0$}   & 
        \makebox[\widthof{$\ket{e,n}$}]{$0$}  & 
        -i e^{-i (\phi+ \pi/2)} \sin \lp \eta\beta/2 \rp    \\
      %
        0   & 
        \makebox[\widthof{$\ket{g,n+1}$}]{$1$}   & 
        0 & 
        0 \\
      %
        0 & 
        \makebox[\widthof{$\ket{g,n+1}$}]{$0$}   & 
        \makebox[\widthof{$\ket{e,n}$}]{$1$} & 
        0 \\
      %
        -i e^{i (\phi+ \pi/2)} \sin \lp \eta\beta/2 \rp  & 
        \makebox[\widthof{$\ket{g,n+1}$}]{$0$}   & 
        \makebox[\widthof{$\ket{e,n}$}]{$0$} & 
        \cos \lp \eta\beta/2 \rp  
    \end{bmatrix} &
    \begin{matrix} \ket{g,n} \\ \ket{g,n+1} \\ \ket{e,n} \\ \ket{e,n+1} \end{matrix} 
\end{matrix}
\end{equation*}
donde $\phi$ es la \textbf{fase} del pulso láser, $\eta$ el parametro de Lamd-Dicke y $\beta = \Omega t$, siendo $\Omega$ la intensidad de acoplamiento (un parámetro) y $t$ el \textbf{tiempo} de aplicación del pulso láser. Para nuestro caso, $n = 0$.

		\SubsubiIt{Red sideband}

Para este caso tenemos 
\begin{equation*} \label{ec_ions_U_rsb}
 \begin{matrix}
    & 
    %
    \begin{matrix}
      \ket{g,n}  & 
      \makebox[\widthof{$-i e^{i (\phi+ \pi/2)} \sin \lp \eta\beta/2 \rp$}]{$\ket{g,n+1}$}  & 
      \makebox[\widthof{$-i e^{-i (\phi+ \pi/2)} \sin \lp \eta\beta/2 \rp$}]{$\ket{e,n}$}   &
      \ket{e,n+1}
    \end{matrix} 	&  \\[.5\normalbaselineskip]
  %
    \mathcal{U}_I^{(\text{rsb})} = & 
    \begin{bmatrix}
        \makebox[\widthof{$\ket{g,n}$}]{$1$}  & 
        0 & 
        0 & 
        \makebox[\widthof{$\ket{e,n+1}$}]{$0$} \\
      %
        \makebox[\widthof{$\ket{g,n}$}]{$0$}   & 
        \cos \lp \eta\beta/2 \rp    & 
        -i e^{-i (\phi+ \pi/2)} \sin \lp \eta\beta/2 \rp & 
        \makebox[\widthof{$\ket{e,n+1}$}]{$0$} \\
      %
        \makebox[\widthof{$\ket{g,n}$}]{$0$} & 
        -i e^{i (\phi+ \pi/2)} \sin \lp \eta\beta/2 \rp   & 
        \cos \lp \eta\beta/2 \rp & 
        \makebox[\widthof{$\ket{e,n+1}$}]{$0$} \\
      %
        \makebox[\widthof{$\ket{g,n}$}]{$0$} & 
        0 & 
        0 & 
        \makebox[\widthof{$\ket{e,n+1}$}]{$1$} 
    \end{bmatrix} &
    \begin{matrix} \ket{g,n} \\ \ket{g,n+1} \\ \ket{e,n} \\ \ket{e,n+1} \end{matrix} 
\end{matrix}
\end{equation*}
donde $\phi$ es la \textbf{fase} del pulso láser, $\eta$ el parametro de Lamd-Dicke y $\beta = \Omega t$, siendo $\Omega$ la intensidad de acoplamiento (un parámetro) y $t$ el \textbf{tiempo} de aplicación del pulso láser. Para nuestro caso, $n = 0$. 


\begin{equation*} \label{ec_ions_CZ}
 \begin{matrix}
    & 
    %
    \begin{matrix}
      \ket{gg} & 
      \ket{ge} & 
      \ket{eg} &
      \ket{ee}
    \end{matrix} 	&  \\[.5\normalbaselineskip]
  %
    \text{CZ} = & 
    \begin{bmatrix}
        \makebox[\widthof{$\ket{gg}$}]{$1$} & 
        \makebox[\widthof{$\ket{gg}$}]{$0$} & 
        \makebox[\widthof{$\ket{gg}$}]{$0$} & 
        \makebox[\widthof{$\ket{gg}$}]{$0$} \\
      %
        \makebox[\widthof{$\ket{gg}$}]{$0$} & 
        \makebox[\widthof{$\ket{gg}$}]{$1$} & 
        \makebox[\widthof{$\ket{gg}$}]{$0$} & 
        \makebox[\widthof{$\ket{gg}$}]{$0$} \\
      %
        \makebox[\widthof{$\ket{gg}$}]{$0$} & 
        \makebox[\widthof{$\ket{gg}$}]{$0$} & 
        \makebox[\widthof{$\ket{gg}$}]{$1$} & 
        \makebox[\widthof{$\ket{gg}$}]{$0$} \\
      %
        \makebox[\widthof{$\ket{gg}$}]{$0$} & 
        \makebox[\widthof{$\ket{gg}$}]{$0$} & 
        \makebox[\widthof{$\ket{gg}$}]{$0$} & 
        \makebox[\widthof{$\ket{gg}$}]{$-1$} \\
    \end{bmatrix} &
    \begin{matrix} \ket{gg} \\ \ket{ge} \\ \ket{eg} \\ \ket{ee} \end{matrix} 
\end{matrix} \,
\end{equation*}





\begin{equation*} \label{ec_ions_SWAPav}
 \begin{matrix}
    & 
    %
    \begin{matrix}
      \ket{g,0} & 
      \ket{g,1} & 
      \ket{e,0} &
      \ket{e,1}
    \end{matrix} 	&  \\[.5\normalbaselineskip]
  %
    \text{SWAP}_{av} = & 
    \begin{bmatrix}
        \makebox[\widthof{$\ket{g,0}$}]{$1$} & 
        \makebox[\widthof{$\ket{g,0}$}]{$0$} & 
        \makebox[\widthof{$\ket{g,0}$}]{$0$} & 
        \makebox[\widthof{$\ket{g,0}$}]{$0$} \\
      %
        \makebox[\widthof{$\ket{g,0}$}]{$0$} & 
        \makebox[\widthof{$\ket{g,0}$}]{$0$} & 
        \makebox[\widthof{$\ket{g,0}$}]{$1$} & 
        \makebox[\widthof{$\ket{g,0}$}]{$0$} \\
      %
        \makebox[\widthof{$\ket{g,0}$}]{$0$} & 
        \makebox[\widthof{$\ket{g,0}$}]{$-1$} & 
        \makebox[\widthof{$\ket{g,0}$}]{$0$} & 
        \makebox[\widthof{$\ket{g,0}$}]{$0$} \\
      %
        \makebox[\widthof{$\ket{g,0}$}]{$0$} & 
        \makebox[\widthof{$\ket{g,0}$}]{$0$} & 
        \makebox[\widthof{$\ket{g,0}$}]{$0$} & 
        \makebox[\widthof{$\ket{g,0}$}]{$1$} \\
    \end{bmatrix} &
    \begin{matrix} \ket{gg} \\ \ket{ge} \\ \ket{eg} \\ \ket{ee} \end{matrix} 
\end{matrix} \,,
\end{equation*}


\begin{equation*} \label{ec_ions_CZav}
 \begin{matrix}
    & 
    %
    \begin{matrix}
      \ket{g,0} & 
      \ket{g,1} & 
      \ket{e,0} &
      \ket{e,1}
    \end{matrix} 	&  \\[.5\normalbaselineskip]
  %
    \text{CZ}_{av} = & 
    \begin{bmatrix}
        \makebox[\widthof{$\ket{g,0}$}]{$1$} & 
        \makebox[\widthof{$\ket{g,0}$}]{$0$} & 
        \makebox[\widthof{$\ket{g,0}$}]{$0$} & 
        \makebox[\widthof{$\ket{g,0}$}]{$0$} \\
      %
        \makebox[\widthof{$\ket{g,0}$}]{$0$} & 
        \makebox[\widthof{$\ket{g,0}$}]{$1$} & 
        \makebox[\widthof{$\ket{g,0}$}]{$0$} & 
        \makebox[\widthof{$\ket{g,0}$}]{$0$} \\
      %
        \makebox[\widthof{$\ket{g,0}$}]{$0$} & 
        \makebox[\widthof{$\ket{g,0}$}]{$1$} & 
        \makebox[\widthof{$\ket{g,0}$}]{$0$} & 
        \makebox[\widthof{$\ket{g,0}$}]{$0$} \\
      %
        \makebox[\widthof{$\ket{g,0}$}]{$0$} & 
        \makebox[\widthof{$\ket{g,0}$}]{$0$} & 
        \makebox[\widthof{$\ket{g,0}$}]{$0$} & 
        \makebox[\widthof{$\ket{g,0}$}]{$-1$} \\
    \end{bmatrix} &
    \begin{matrix} \ket{gg} \\ \ket{ge} \\ \ket{eg} \\ \ket{ee} \end{matrix} 
\end{matrix} \,,
\end{equation*}

\end{document}